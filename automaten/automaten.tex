\section{Automaten}
\rule{\textwidth}{0.4pt}
\newpage
\subsection{Endlicher Automat}
    Ein endlicher Automat, EA, ist ein Tupel \(A=(Q,\Sigma,\Delta,s,F)\).
    \begin{itemize}
        \item Q ist eine endliche Menge, die Zustandsmenge 
        \item \(\Sigma\) das Eingabealphabet
        \item \(\Delta\subseteq Q\times\Sigma\times Q \) die Übergangsrelation, so dass es 
        für alle \(q\in Q\) und \(a\in\Sigma\) ein \(q'\in Q\) mit \((q,a,q')\)
        \item \(s\in Q\) der Startzustand 
        \item \(F\subseteq Q\) die Menge der akzeptierenden Zustände
    \end{itemize}
    \textit{Der endliche Automat A ist ein deterministischer endlicher Automat, kurz DEA, wenn es
    \(\forall(q,a)\in Q\times\Sigma\) genau ein q' gibt mit \((q,a,q')\in\Delta\).
    Im Sinne der obigen Betrachtung entspricht ein EA \(A=(Q,\Sigma,\Delta,s,F)\) der 1-TM
    \(M_A=(Q,\Sigma,\Sigma\cup\{\square\},\{(q,a,q',a,R):(q,a,q')\in\Delta\},s,F)\).} 
\subsection{Übergangsfunktion eines EA}
    Die Übergangsfunktion eines EA A ist die Funktion \(\delta _A^*(q,\lambda)=\{q\}\) und
    \[\delta_A^*(q,aw)=\bigcup\limits_{q'\in\delta_A(q,a)}\delta_A^*(q',w)\] 
    \(\forall q\in Q,a\in\Sigma, w\in\Sigma^*\)\\
    Für \(Q_0\subseteq Q\) und \(w\in\Sigma^*\) schreiben wir \(\delta_A^*(Q_0,w)\) statt 
    \(\bigcup\limits_{q\in Q_0}\delta_A^*(q,w)\).\par\bigskip
    \textit{Sei A = (Q, $\Sigma$, $\Delta$, s, F) ein EA
    \begin{itemize}
        \item [(i)] $\forall$ q $\in$ Q und a$\in \Sigma$ gilt $\delta_{A}^{*}$(q,a) = $\delta_{A}$(q, a).
        \item [(ii)] Ist A ein DEA, q $\in$ Q, a $\in \Sigma$ und w $\in \Sigma^{*}$, und $\lvert \delta_{A}^{*}(q,w) \rvert$ = 1??.
        \item[(iii)] Seien u,v $\in \Sigma^{*}$ $\forall$ q $\in$ Q gilt $\delta_{A}^{*}$(q, uv) = $\delta_{A}^{*}(\delta_{A}^{*}(Q_{0}, u), v)$.
    \end{itemize}}
\subsection{Übergangsfunktion eines DEA}
    Sei A ein DEA. Auch die Funktion \(\delta_{det,A}:Q\times\Sigma\to Q\) mit \(\delta_a(q,a)=\{\delta_{drt,A}(q,a)\}\)
    \(\forall q\in Q\) und \(a\in \Sigma\) wird Übergangsfunktion von A genannt.
    Analoges gilt für \(\delta_{det,A}^*\) und \(\delta_A^*\).\\
    Für \(Q_0\subseteq Q\) und \(w\in\Sigma^*\) schreiben wir auch \(\delta_{det,A}^*(Q_0,w)\).\par\bigskip
    \textit{Sei Q eine endliche Menge, $\Sigma$ ein Alphabet, s$\in$ Q, und F$\subseteq$ Q. 
    \begin{itemize}
        \item [(i)] $\forall$ Funktionen $\delta$ : Q $\times \Sigma \rightarrow 2^{Q}$ gibt es genau einen EA A = (Q, $\Sigma$, $\Delta$, s, F) mit $\delta_{A}$ = $\delta$.
        \item [(ii)] $\forall$ Funktionen $\delta$ : Q $\times \Sigma \rightarrow$ Q gibt es genau einen $\delta_{det, A}$ = $\delta$. 
    \end{itemize}}
\subsection{akzeptierte Sprache}
    Sei A ein EA. Die Sprache \(L(A):=\{w\in\Sigma^*:\delta_A^*(s,w)\cap F\neq\varnothing\}\) ist die akzeptierte Sprache von A. 
\subsection{Regulär}
    Eine Sprache L heißt regulär, wenn es einen EA A mit L(A)=L gibt.\\
    Wir schreiben REG für die Klasse der regulären Sprachen.
    \subsubsection*{Beispiel}
    \textit{Zu jedem Zeitpunkt während der Verbindung der Eingabe durch einen endlichen Automaten höngt der restliche Bearbeitung immer nur vom gegewärtigen Zustand und dem noch einzulesenden Teil der Eingabe ab, nicht aber wie bei TM im allgemeinen von vergangenen Bandmanipulation. Interpretiert man die Eingabe als von einer äußeren Quelle kommend, so ist der  Zustand des Automaten also allein durch seinen Zustand gegeben und der nächste Zustand hängt nur vom Zugeführten Symbol ab. Daher bietet sich eine Darstellung eines EA durch ein Übergangsdiagramm oder eine sogenannte Übergangstabelle an.}\\
    Sei A := (\{$q_{0}$, $q_{1}$\}, \{0, 1\}, $\Delta$, $q_{0}$, \{$q_{1}$\}) mit $\Delta$ = \{($q_{0}$)\}
Übergangsdiagramm und übergangstabelle von sehen wie folgt aus:
\begin{center}
    \begin{tabular}{|c|c|c|}
        \hline
        Zustand/Symbol & 0 & 1 \\
        \hline
        $q_{0}$ & $q_{0}$ & $q_{1}$ \\
        \hline
        $q_{1}$, * & $q_{1}$ & $q_{0}$ \\
        \hline
    \end{tabular}            
\end{center}
\subsection{Übergangsdiagramm}
    Für jeden Zustand gibt es einen Kreis. Zustände in F bekommen einen Doppelkreis. Für \((q,a,q')\in\Delta\) für wir 
    einen Pfeil von dem Kreis von q zu dem Kreis von q' mit der Beschriftung a.\\
    Zustätzlich gibt es einen Pfeil (ohne Beschriftung) aus dem 'Nichts' zu dem Kreis des Startzustandes.
\subsection{Potenzautomat}
    Sei A ein EA. Der Potenzautomat von A ist der DEA \(P_a=(2^Q, \Sigma, \Delta',\{s\},\{P\subseteq Q:P\cap F\neq\varnothing\})\) mit
    \[\delta_{det,P_a}(Q_0,a)=\bigcup\limits_{q\in Q_0}\delta_A(q,a)\text{ }\forall Q_0\subseteq Q\text{ }\forall a\in\Sigma\]
    \subsubsection*{Satz}
        Eine Sprache L ist genau dann regulär, wenn es einen DEA A mit L(A) = L gibt.\par\bigskip
        \begin{proof}
            Sei A = (Q, $\Sigma$, $\Delta$, s, F) ein EA mit Potenzautomat $P_{A}$. Es genügt zu zeigen, dass L(A) = L($P_{A}$). Hierfür genügt es zu zeigen, dass:
\[\delta_{det,P}^{*} = \delta_{A}^{*}(s, w) \forall w \in \Sigma^{*} (*)\]
Denn damit folgt
\[w \in  L (P_{A}) \Leftrightarrow \delta_{P_{A}}^{*}(\{s\}, w) \cap \{P\subseteq Q : P\cap F \neq \varnothing \} \neq \varnothing \] 
\[\Leftrightarrow \delta_{det, P_{A}}^{*}(\{s\}, w) \cap F \neq \varnothing \]
\[\underset{\text{(*)}}{\Leftrightarrow } \delta_{A}^{*}(\{s\}, w) \cap F \neq \varnothing \]
\[\Leftrightarrow w \in L(A)\] Wir zeigen (*) mittels vollständiger Induktion über $\lvert$w$\rvert$. Es gilt $\delta_{det, P_{A}}^{*}$(\{s\}, $\lambda$) = $\delta_{A}^{*}$(s, $\lambda$). Sei w $\in \Sigma^{+}$ mit $\delta_{det, A}^{*}$(\{s\}, v) = $\delta_{A}^{*}$(s, v) $\forall$ v $\in \Sigma^{\leq \lvert w \rvert - 1}$. Nun zeigen wir (*) Sei va := w mit a $\in \Sigma$ und $\lvert$v$\rvert$ = $\lvert$w$\rvert$ - 1.
\[\delta_{det, P_{A}}^{*} (\{s\}, w) \underset{\text{Bem 4.5}}{=} \delta_{det, P_{A}}^{*} (\delta_{det, P_{A}}^{*}(\{s\}, v), a)\]
\[\underset{\text{Ind. hyp}}{=} \delta_{det, P_{A}}^{*}(\delta_{det, P_{A}}^{*}(\{s\}, v), a)\]
\[ = \bigcup \limits_{q \in \delta_{det, A}^{*}}\delta_{A}(q, a)\]
\[ = \delta_{A}^{*}(\delta_{A}^{*}(s, v), a)\]
\[ = \delta_{A}^{*}(s, va)\]
\[ = \delta_{A}^{*}(s, w)\]
        \end{proof}