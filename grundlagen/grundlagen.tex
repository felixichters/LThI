\section{Grundlagen}
\rule{\textwidth}{0.4pt}
\subsection{Alphabet}
    Ein Alphabet ist eine nichtleere Menge \(\Sigma\),die Elemente heißen Symbole.
\subsection{Wörter}
    Ein Wort über einem Alphabet \(\Sigma\) ist eine endliche Folge von Symbolen aus \(\Sigma\).\\
    Für \(i\in[|w|]\) bezeichnet \(w(i)\) das i-te Element von w und für Symbole \(a_1,\dots,a_n\in \Sigma\)
    bezeichnet \(a_1,a_2,\dots,a_n\) das Wort w der Länge n mit \(w(i)=a_i\) \(\forall i\in [n]\).
\subsection{Sprache}
    Eine Sprache ist eine Menge von Wörtern über einem gemeinsamen Alphabet \(\Sigma\).
\subsection{\(\Sigma^*\)}
    Die Menge aller Wörter über \(\Sigma\) wird mit \(\Sigma^*\) bezeichnet.\\
    Für \(n\in\mathbb{N}_0\) setzen wir
    \[\Sigma^{\leq n}:=\{w\in\Sigma^*:|w|\leq n\}\]
    \[\Sigma^{=n}:=\{w\in\Sigma^*:|w|=n\}\]
    \[\Sigma^{\geq n}:=\{w\in\Sigma^*:|w|\geq n\}\]
    \[\Sigma^{+}:=\Sigma^{\geq 1}\]
\subsection{Verkettung}
    Für Wörter \(w_1,w_2\) ist die Verkettung \(w_1w_2\) ist definiert durch 
    \[w_1w_2:=w_1(1)\dots w_1(|w_1|)w_2(1)\dots w_2(|w_2|).\]
    Für Sprachen \(L_1,L_2\) sei \(L_1L_2\) definiert durch 
    \[L_1L_2:=\{w_1w_2:w_1\in L_1, w_2\in L_2\}\]
\subsection{Präfix, Infix, Suffix}
    Seien u,v Wörter 
    \begin{itemize}
        \item u ist Präfix von v, kurz \(u\sqsubseteq v\), falls es ein Wort w gibt, sodass uw=v 
        \item u ist Infix von v, falls es Wörter \(w_1,w_2\) gibt, sodass \(v=w_1uw_2\)
        \item u ist Suffix von v, falls es ein Wort w gibt, sodass v=wu
    \end{itemize}
\subsection{Homomorphismus}
    Für Sprachen \(L,M\) heißt eine Funktion \(\varphi:L\to M\) 
    Homomorphismus von Sprachen, wenn gilt: \[\varphi(uv) = \varphi(u)\varphi(v)\]
\subsection{Längenlexikographische Ordnungen}
    Es gilt \(u \leq_{llex} v\) wenn eine der folgenden Bedingungen erfüllt ist:
    \begin{enumerate}
        \item \(|u|<|v|\)
        \item \(|u|=|v|\) und ist \(i\in[|u|]\) minimal mit \(u(i)\ne v(i)\), so gilt \(u(i)\leq v(i)\)
    \end{enumerate}
\subsection{bin(i)}
\(bin(i)\) ist die Funktion für das in Längenlexikopraphischer Reihnfolge \(i+1\)-te Binärwort.\\
\(1bin(i)\) beschreibt die Binärdarstellung von \(i+1\).