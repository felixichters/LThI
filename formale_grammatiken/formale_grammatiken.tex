\section{Formale Grammatiken}
\fancyhead[R]{}
\fancyhead[C]{Formale Grammatiken}
\fancyhead[L]{Definitionen}
\paragraph*{Idee:} Konstruktion aller Wörter einer Sprache.
\paragraph*{Beispiel:} \(L=\{0^n\}_{n\in\mathbb{N}_0}\)\\
S Startsymbol\\
\(S\to X,S\to 0S\) Regel 
\(S\to 0S\to 00S\to 000S\to 000\)
\subsection{Grammatiken}
    Eine Grammatik ist ien Tupel \(G=(N,T,P,S)\). Dabei ist 
    \begin{itemize}
        \item N das Alphabet der Nichtterminalsymbole/Variablen 
        \item T das Alphabet der Terminalsymbole mit \(N\cup T=\varnothing \)
        \item \(P\subseteq((n\cup T)^*\backslash T^*)\times (N\cup T)^*\) eine endliche Menge von Regeln
        \item \(S\in\mathbb{N}\) das Startsymbol
    \end{itemize}
    EIne Satzform von G ist ein Wort \(s\in(N\cup T)^*\) und eine Terminalwort von G ist ein Wort \(t\in T^*\).
\subsection{Ableitung}
    Sei \(G=(N,T,P,S)\) eine Grammatik. Eine Satzform \(w'\) von G ist in einem Schritt aus einer Satzform w von G ableitbar, wenn es Satzformen u,v,x,y von G gibt, so dass \(w=xuy,u\to v\in P\) und \(w'=xvy\) gelten. Es bezeichne \(\to_G\) die Relation auf der Menge der Satzformen von G, sodass \(w\to_Gw'\) genau dann für Satzformen von G gilt, wenn w' aus w in einem Schritt ableitbar ist.\\
    Für Satzformen u,v von G ist eine Ableitung von v aus u eine Folge \(u=w_1,\dots,w_n=v\) mit \(w_i\to_Gw_{i+1}\) \(\forall i\in[n-1]\) und eine Ableitung von v in G ist eine Ableitung von v aus S in G. Für \(n\in\mathbb{N}\) schreiben wir \(u\to^n_Gv\) wenn es eine Ableitung von v aus u der Länge n gibt und wir schreiben \(u\to^*_Gv\) wenn eine Ableitung von v aus u in G existiert.
\subsection{Erzeugte Sprache}
    Sei \(G=N,T,P,S\) eine Grammatik. Die von G erzeugte Sprache \(L(G)\) ist die Menge aller Wörter \(w\in T^*\) für die es eine Ableitung von w in G gibt.
\paragraph*{Lemma} 
    Sei \(G=(N,T,P,S)\) eine Grammatik und seien u,v,x,y Satzformen von G und seien \(n,m\in\mathbb{N}\) mit \(u\to_G^nv\) und \(w\to_u^nxuy\).\\
    Dann gilt \(w\to_u^{m+n-1}xvy\).
\paragraph*{Beweis}
    Sei \(\alpha _1,\dots,\alpha_n\) ein Ableitung von xuy aus w und \(\beta_1,\dots,\beta_m\) eine Ableitung von v aus u. Dann ist\\ 
    \(\alpha_1,\dots,\alpha_{n-1},x\beta_1y,\dots,x\beta_my=xvy\)\\
    eine Ableitung von xvy aus w in G der Länge n+m-1.\par\bigskip
    Im folgenden beschäftigen wir uns mit dem Thema welche Sprache Grammatiken verschiedener Komplexitätsstufen erzeugen können.
\paragraph*{Satz}
    Eine Sprache ist genau dann rekuriv aufzählbar, wenn sie von einer Grammatik erzeugt wird.
\paragraph*{Beweisidee} Wird eine Sprache L von einer Grammatik erzeugt, so ist L die erkannte Sprache einer TM, die in geeigneter Weise Ableitungen von G erzeugt, prüft ob diese Ableitung dem Wort der Eingabe entspricht und gegebenfalss akzeptiert. Wenn eine Ableitung der Eingabe gefunden ist.\par\bigskip
Gegebn eine rekursiv aufzählbare Sprace L und eine TM, die L erkennt. So konstruieren wir ähnlich dem Postschen Korrespondenzproblems Regeln und Symobole, sodass wir die Arbeitsweise der TM modellieren können und entsprechend mit einem Terminalwort enden wenn dies von der TM erkannt wird.
\subsection{Rechtslinear}
    Eine Grammatik \(G=(N,T,P,S)\) ist rechtslinear, wenn alle Regeln von der Form 
    \[X\in uy \text{ oder } X\to u\]
    mit \(X,y\in\mathbb{N}\) und \(u\in T^*\) sind.\par\bigskip 
    Hier ist es sinnvoll endliche Automaten zu betrachten bei denen es nicht \(\forall\) Zustände q und Eingabesymbole a ein Tripel \((q,a,q')\) in der Übergangsrelation geben muss. Solche Automaten sind zwangsläufig nicht deterministisch.
\paragraph*{Satz}
Eine Sprache ist genu dann regulär, wenn sie von einer rechtslinearen Grammatik erzeugt wird.
\paragraph*{Beweisidee}
Zunächst überzeugt man sich davon, dass eine Sprache L genau dann von einer rechtlinearen Grammatik erzeugt wird, wenn sie von einer Grammatik \(G=(N,T,P,S)\) erzeugt wird bei der alle Regeln von der Form \(X\to ay\) oder \(X\to \lambda\)\\
mit \(x,y\in\mathbb{N}\) und \(a\in T\) sind. Eine Solche Grammatik wird als Grammatik in Simulationsform bezeichnet.\par\bigskip
Die Sprache L die von einer rechtslinearen Grammatik (in Simulationsform) gebildet wird von dem EA 
\[A=(N,T,\Delta,S,\{X\in N: X\to \lambda\in P\})\]
mit 
\[\Delta=\{(X,a,y)\in N\times T\times N:X\to ay\in P\}\]
erkannt.\\
Umgekehrt ist es einfach zu sehen, dass jede Reguläre Sprace von einer rechtslinearen Grammatik erzeugt wird.