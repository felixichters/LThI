\section{Turingmaschienen} 
\rule{\textwidth}{0.4pt}
\newpage
\subsection{Turingmaschiene}
    Eine k-Band Turingmaschiene ist ein Tupel \[M=(Q,\Sigma,\Gamma,\Delta,s,F)\]
    \begin{itemize}
        \item \(Q\) ist die endliche Zustandsmenge
        \item \(\Sigma\) ist das Eingabealphabet
        \item \(\Gamma\) das Bandalphabet mit \(\Sigma\subseteq\Gamma\) und \(\square\in\Gamma\backslash\Sigma\)
        \item \(\Delta\subseteq Q\times\Gamma^k\times Q\times\Gamma^k\times {\{L,S,R\}}^k\) die Übergangsrelation
        \item \(s\in Q\) der Startzustand
        \item \(F\subseteq Q\) die Menge der akzeptierten Zustände
    \end{itemize}
    Die Elemente von \(\Delta\) heißen Instruktionen, eine Instruktion sieht wie folgt aus:
    \[(q,a_1,\dots,a_k,q',a'_1,\dots,a'_k,B_1,\dots,B_k)\]
    \textit{Eine TM ist eine DTM, wenn es \(\forall b\in Q\times\Gamma^k\) höchstens eine Instruktion \(i\in \Delta\) mit Bedingungsteil b gibt}
\subsection{Konfiguration}
    Eine Konfiguration einer k-TM ist ein Tupel 
    \[C=(q,w_1,\dots,w_k,p_1,\dots,p_k)\in Q \times (\Gamma^*)^k \times \mathbb{N}^k\]
    Die Startkonfiguration zur Eingabe \((u_1,\dots,u_n)\in(\Sigma^*)^n\) mit \(n\in\mathbb{N}\), ist die Konfiguration
    \[start(u_1,\dots,u_n)=(s,u_1\square u_2\square\dots\square u_n,\square,\dots,\square,1,\dots,1)\]
    \textit{Die Stoppkonfiguartion ist die Konfiguration zu der es keine Nachfolgekonfiguration gibt}
\subsection{Nachfolgekonfiguration}
    Für Konfigurationen \(C=(q,w_1,\dots,w_k,p_1,\dots,p_k)\) und \(C'=(q',w'_1,\dots,w'_k,p'_1,\dots,p'_k)\) einer k-TM, \\
    ist die Konfiguration \(C'\) \textbf{Nachfolgekonfiguration} von C, wenn es eine Instruktion 
    \[(q,w_1(p_1),\dots,w_k(p_k),q',a'_1,\dots,a'_k,B_1,\dots,B_k)\in\Delta\] 
    gibt sodass
    \[
    w_i = \begin{cases}
        \square a'_iw_i(2)\dots w_i(|w_i|) & \text{ falls } p_i=1 \text{ und } B_i=L\\
        w_i(1)\dots w_i(|w_i|-1)a'_i\square & \text{ falls } p_i=|w_i| \text{ und } B_i = R\\
        w_i(1)\dots w_i(p_1-1)a'_iw_i(p_i+1)\dots w_i(|w_i|) & \text{ sonst}
    \end{cases}    
    \]
    und 
    \[
    w_i = \begin{cases}
        1 & \text{ falls } p_i=1 \text{ und } B_i=L\\
        p_i-1 & \text{ falls } p_i \geq 2 \text{ und } B_i = L\\
        p_i & \text{ falls } B_i=S\\
        p_i+1 & \text{ falls } B_i=R
    \end{cases}    
    \]
    für alle \(i\in [k]\) gelten.
\subsection{Rechnung}
    \textit{Es bezeichne \(\to _M\) die Relation auf der Menge der Konfigurationen einer k-TM M, sodass \(C\to _M C'\) \\
    falls \(C,C'\) Konfigurationen von M sind wobei C' eine Nachfolgekonfiguration von C ist.} \par\bigskip
    Eine \textbf{endliche partielle Rechnung} ist eine endliche Folge \(C_1,\dots,C_n\) von Konfigurationen von 
    M mit \(C_i\to _M C_{i+1}\forall i\in[n-1]\).\\
    Eine \textbf{unendlich partielle Rechnung} ist eine unendliche Folge \(C_1,C_2,\dots\) von Konfigurationen von 
    M mit \(C_i\to _M C_{i+1}\forall i+1\in\mathbb{N}\).
    Eine Rechnung zur Eingabe \((w_1,\dots,w_n)\in(\Sigma^*)^n\) mit \(n\in \mathbb{N}\) ist ein 
    unendlich partielle Rechnung \(start_M(w_1,\dots,w_n)=C_1,C_2,\dots,C_m\).
    \par\bigskip
    \textit{Ist M eine k-DTM, so gilt es \(\forall n \in \mathbb{N}\) und \((w_1, \cdots, w_n) \in (\Sigma^*)^n\) genau eine Rechnung zur Eingabe \((w_1, \cdots, w_n)\).}
    Ist M eine k-DTM, so gilt es \(\forall n \in \mathbb{N}\) und \((w_1, \cdots, w_n) \in (\Sigma^*)^n\) genau eine Rechnung zur Eingabe \(w_1, \cdots, w_n)\).
\subsection{Total}
    Totale TMs, sind TMs die bei jeder Eingabe immer anhalten. Alle Rechnungen müssen endlich sein.
\subsection{Akzeptierte Sprache}
    \textit{Eine Stoppkonfiguration ist akzeptiert, wenn \(q\in F\).}\par\bigskip
    Die akzeptierte Sprache L(M) ist die Sprache über dem Alphabet \(\Sigma\), so dass \(\forall w\in\Sigma^*\) genau dann 
    \(w\in L(M)\) gilt, wenn es eine endliche Rechnung \(C_1,\dots,C_n\) zur Eingabe w gibt, bei der \(C_n\) eine 
    akzeptierte Stoppkonfiguartion ist.\par\bigskip
    \textit{Für nicht-deterministische Tms heißt das, dass es für die Wörter in der akzeptierten Sprache nur mind. eine in einer 
    akzeptierten Stoppkonfiguartion endende endliche Rechnungen zur Eingabe w geben muss.\\
    Für Wörter w die nicht in L(M) sind, sind alle Rechnung von M zur Eingabe am Ende nicht in einer akzeptierten Stoppkonfiguartion oder unendlich}
\subsection{Entscheidbar}
    Eine Sprache L ist genau dann entscheidbar, wenn es eine totale k-TM mit L(M) = L gibt.
\subsection{Rekursiv aufzählbar}
    Eine Sprache L ist genau dann rekursiv aufzählbar, wenn es eine k-TM mit akzeptierter Sprache L gibt.\par\bigskip
    \textit{Wir schreiben RE für die Klasse der rekursiv aufzählbaren Sprachen. Die Aufzählbarkeit leitet sih daraus ab, dass es für eine rekuriv aufzählbare Sprache L über einem Alühabet $\Sigma$ möglich ist effektive Verfahren anzugeben ,die die Wörter von L aufzählen, also dass eine endlich oder unendliche Aufzählung von $A = w_1, w_2, \cdots$ mit ${w_1, w_2, \cdots} = L$ existiert.
    \begin{itemize}
        \item Jede entscheidbare Sprache ist rekuriv aufzählbar
        \item Alle endlichen Sprachen sind entscheidbar 
        \item Eine Sprache L über einem Alphabet $\Sigma$ist genau dann entscheidbar, wenn $L$ und $L^c :=(\Sigma^*)/L$ rekursiv aufzähbar sind.
    \end{itemize}
    }
\subsection{Ausgabe} 
    Die Ausgabe \(out_M(C)\) bei Konfiguration C ist das Präfix \(w\sqsubseteq w_1(p_1)\dots w_1(|w_1|)\) max. Länge mit 
    \(w\in (\Gamma\backslash\{\square\})^*\).\par\bigskip
    \textit{Also das längst mögliche Wort, dass auf Band 1 rechts vom Lesekopf steht und nicht \(\square\) enthält.}
\subsection{Berechnete Funktion}
    Sei M eine k-DTM. Die von M berechnete n-äre partielle Funktion \(\varphi_M\) ist die partielle Funktion:
    \[\varphi_M:(\Sigma^*)^n\rightsquigarrow(\Gamma\backslash\{\square\})^*\]
    sodass folgendes gilt:
    \begin{enumerate}
        \item Ist die Rechnung zur Eingabe \(w_1,\dots,w_n\) die endliche Rechnung \(C_1,\dots,C_m\) so gilt\\ 
        \(\varphi_M(w_1,\dots,w_n)=out_M(C_m)\)
        \item Ist die Rechnung zur Eingabe \(w_1,\dots,w_n\) unendlich, so gilt \(\varphi_M(w_1,\dots,w_n)\uparrow\)
    \end{enumerate}\bigskip
    \textit{Nur deterministische TMs können Funktionen berechnen.}
\subsection{Partiell Berechnbar}
    Für \(\Sigma,\Gamma\) und eine partielle Funktion \(U:\Sigma^*\rightsquigarrow\Gamma^*\) ist \(\varphi\) berechenbar, 
    wenn es ein \(k\in\mathbb{N}\) gibt und eine k-DTM M mit \(\varphi_M=\varphi\) gibt.\\
    Ist \(\varphi\) total und partiell berechenbar, so ist \(\varphi\) berechenbar.
\subsection{Charakteristische Funktion}
    Sei L eine Sprache über dem Alphabet \(\Sigma\).
    \begin{enumerate}
        \item Die charakteristische Funktion von L als Sprache über \(\Sigma\) ist die Funktion\\
        \(\mathbbm{1}_L:\Sigma^*\to\{0,1\}\) mit \(\mathbbm{1}_L(w)=1\) \(\forall w\in L\) und \(\mathbbm{1}_L(w)=0\) \(\forall w\in\Sigma^*\backslash L\).
        \item Die partiell charakteristische Funktion von L als Sprache über \(\Sigma\) ist die partielle Funktion\\
        \(\chi_L:\Sigma^*\rightsquigarrow\{1\}\) mit \(\chi_L(w)=1\) \(\forall w\in L\) und \(\chi_L(w)\uparrow\) \(\forall w\in\Sigma^*\backslash L\).
    \end{enumerate}
    \subsubsection*{Bemerkung}
        Sei L eine Sprache über einem Alphabet \(\Sigma\)
        \begin{itemize}
            \item L ist genau dann entscheidbar, wenn $\mathbbm{1}_L$ berechenbar ist.
            \item L ist genau dann rekursiv aufzähbar, wenn $x_L$ partiell berechenbar ist.
        \end{itemize}
\subsection{Normiet}
    Eine 1-DTM M heißt normiert, wenn Q={0,...,n} für ein \(n\in\mathbb{N}_0\), \(\Sigma=\{0,1\}\), \(\Gamma=\{\square,0,1\}\), \(\Delta=0\), \(F=\{s\}\).\par\bigskip
    \textit{
        Alle TMs mit Eingabealphabet {0,1} lassen sich mit folgenden Schritten in eine normierte TM mit gleicher erkannter Sprache und gleicher berechneten Funktion umwandeln.}
    \begin{itemize}
        \item Von nicht-Determinismus zu Determinismus
            \subitem Eine DTM kann die Rechnungen einer nicht-Deterministischen TM parallel im Sinne von abwechselnd schrittweise durchführen 
            um schließlich das Verhalten der simulierten TM zu imitieren.
            Das entspricht einer Breitensuche im Rechnungsbaum.
        \item Von mehreren Bändern zu einem Band
            \subitem Intuitiv können k Bänder auf einem Band simuliert werden, indem die Felder des einen Bandes in k-Teilfelder unterteilt werden, 
            die jeweils die gleichen Bandalphabetbuchstaben wie zuvor als Beschriftung zulassen und es zudem erlauben zu notieren, dass der simulierte Kopf des simulierten Bandes dort steht.  
        \item Von beliebigem Bandalphabet zu \(\{\square,0,1\}\)
        \subitem Andere Bandalphabete können bei einem Alphabet Wechsel zum Bandalphabet \(\{\square,0,1\}\) simuliert werden, 
        indem mehrere nebeneinander liegende Felder verwendet werden um ein Symbol des vorigen Bandalphabets durch ein Binärwort zu beschreiben .
        Die TM liest stets nur ein Feld, es wird daher also nötig sein die Zustandsmenge so zu erweitern, dass angrenzende Felder im Zustand gespeichert werden können.
    \end{itemize}
\subsection{Code}
    Wir betrachten die Funktion code mit geeigneter Definitionsmenge und Zielmenge \(\{0,1\}\).
    \begin{itemize}
        \item Codieren der Bewegungsrichtung
            \subitem \(code(L)=10\)
            \subitem \(code(S)=00\)
            \subitem \(code(R)=01\)
        \item Codieren der einzelnen Instruktionen
            \subitem \(I=(q,a,q',a',B)\)
            \subitem \(code(I)=0^{|bin(q)|}1bin(q)a0^{|bin(q')|}1bin(q')a'code(B)\)
        \item Codieren des Instruktinssatzes 
            \subitem \(code(\Delta)=code_1(\Delta)\dots code_{|\Delta|}(\Delta)\)
        \item Codieren der normierten TM 
            \subitem \(code(M)=0^{|bin(n)|}1bin(n)code(\Delta)\)    
    \end{itemize}
    \textit{Jede normierte TM hat einen Code und zwei verschiedene niemals den gleichen. Die Sprache der TMs ist entscheidbar.}   
\subsection{Standardaufzählung}
    \textit{Sei \(\hat{w}_0,\hat{w}_1,\dots\) die Aufzählung aller Codes normierter TMs in längenlexikopraphischer Ordnung.}\bigskip
    Für \(e\in\mathbb{N}\) sei \(\mathcal{M}_e\) die durch \(\hat{w}_e\) codierte TM und für \(n\in\mathbb{N}\) sei \(\Phi_e^n:\mathbb{N}_0^n\to\mathbb{N}_0\)
    die von \(\mathcal{M}_e\) berechnete n-äre partielle Funktion.\\
    Für \(n\in\mathbb{N}\) heißt die Folge \((\Phi_e^n)_e\in\mathbb{N}\) Standardaufzählung der n-ären partiell berechenbaren Funktion.\\
    Für \(n\in\mathbb{N}\) und eine partiell berechenbare n-äre partielle Funktion \(\varphi:\mathbb{N}_0^n\to\mathbb{N}_0\) heißt jede Zahl 
    \(e\in\mathbb{N}_0\) mit \(\Phi_e^n=\varphi\) Index von \(\varphi\).\par\bigskip
    \textit{Ergibt sich n aus dem Kontext so schreiben wir auch \(\Phi_e\) statt \(\Phi_e^n\)}\par\bigskip
    \textit{Für $n \in \mathbb{N}$ und eine partielle berechnbare n-äre partielle Funktion $\Phi : \mathbb{N}_0^n \rightarrow \mathbb{N}_0$ gibt es unendlich viele Indizes von $\varphi$.}
\subsection{U}
    Es bezeichne \(\mathcal{U}\) die normierte TM, bei Eingabe \((e,x_1,\dots,x_n)\in\mathbb{N}_0^{n+1}\) wobei \(n\in\mathbb{N}\) die normierte TM \(\mathcal{M}_e\)
    bei Eingabe \(x_1,\dots,x_n\) simuliert und falls diese terminiert die Asugabe der Simulation ausgibt.
\subsection{Universell}
    Eine DTM \(\mathcal{U}\) heßt universell, wenn es für alle \(n\in\mathbb{N}\) und alle partiell berechenbaren Funktionen \(\varphi:\mathbb{N}_0^n\rightsquigarrow\mathbb{N}_0\) 
    ein \(e\in\mathbb{N}_0\) gibt, sodass \(\mathcal{U}(e,x_1,\dots,x_n)=\varphi(x_1,\dots,x_n)\) für alle \(x_1,\dots,x_n\in\mathbb{N}_0\) gilt.
\subsection{s m n-Theorem}
    Für alle \(m,n\in\mathbb{N}\) existiert eine berechenbare Funktion \(s_n^m:\mathbb{N}_0^{m+1}\to\mathbb{N}_0\) mit 
    \[\Phi_e^{m+n}(x_1,\dots,x_m,y_1,\dots,y_n)=\Phi^n_{s_n^m(e,x_1,\dots,x_m)}(y_1,\dots,y_n)\]
    für alle \(e,x_1,\dots,x_m,y_1,\dots,y_n\in\mathbb{N}_0\).\par\bigskip
    \begin{proof}
        Fixiere $m \in \mathbb{N}$. Betrachte die DTM S , die bei Eingabe $(e, x_1, \cdots, x_m) \in \mathbb{N}_0^{m+1}$ wie folgt verfährt??.
\begin{itemize}
  \item Zunächst bestimmt S den Code von $\mathcal{M}_e$ 
  \item der Code von $\mathcal{M_e}$wird dann in einen Code einer normierten TM $\mathcal{M}$ umgewandet, die zunächst $x_1\Box\cdots\Box x_m\Box$ neben die Eingabe schreibt, dan den Kopf auf das erste Feld des beschriebenen Bandteilsbewegt und dann wie $\mathcal{M_e}$ arbeitet.
  \item Es wird bestimmt an welcher Stelle der Standardaufzählung der Code von auftaucht und diese Stelle wird ausgegeben.
\end{itemize}
Sei $s_n^m$ die von S berechnete $(m+1)$-äre partielle Funktion. Dann ist $s_n^m$ eine Funktion wie gewünscht.
    \end{proof}
\subsection{Diagonales Halteproblem}
    Die Menge \(H_{diag}:=\{e\in\mathbb{N}_0:\Phi_e(e)\downarrow\}\) heißt diagonales Halteproblem.\\
    \textit{Das diagonale Halteproblem ist rekursiv aufzählbar, jedoch nicht entscheidbar.}
\subsection{m-Reduktion}
    Für eine Sprache \(A\) über einem Alphabet \(\Sigma\) und eine Sprache \(B\) über einem Alphabet \(\Gamma\)
    ist \(A\) genau dann many-one-reduzierbar, auch m-reduzierbar, auf \(B\), kurs \(A\leq_m B\), wenn es eine berechenbare Funktion 
    \(f:\Sigma^*\to\Gamma^*\) gibt, so dass
    \[w\in A\Leftrightarrow f(w)\in B\]
    für alle \(w\in\Sigma^*\) gilt.\\
    \textit{gelten \(A\leq_m B\) und \(B\leq_m A\), so sind A und B m-äquibalent, kurz \(A=_m B\).}
\subsection{Postsches Korrespondenzproblem}
    Für ein Alphabet \(\Sigma\) sei eine Instanz des Postschen Korrespondenzproblem über \(\Sigma\)
    eine endliche Teilmenge \(I\subseteq(\Sigma^+)^2\) Eine Lösung für eine solche Instanz ist eine 
    endliche Folge \((u_1,v_1),\dots,(u_n,v_n)\) von Paaren in I mit \(n\geq1\), so dass 
    \[u_1\dots u_n=v_1\dots v_n\]
    Gibt es eine Lösung für eine Instanz des Postschen Korrespondenzproblems, so heißt diese Instanz lösbar.\\
    Das Postsche Korrespondenzproblem über einem Alphabet \(\Sigma\), kurz \(PCP_\Sigma\), ist die Menge 
    aller lösbarer Instanzen des Postschen Korrespondenzproblems über \(\Sigma\).\par\bigskip
    Für ein Alphabet \(\Sigma\) sei eine Instanz des modifizierten Postschen Korrespondenzproblems über \(\Sigma\)
    ein Paar \((p,I)\), wobei \(I\subseteq(\Sigma^+)^2\) eine endliche Teilmenge und \(p\in I\) ein Paar von Wörtern ist.\\
    Eine Lösung für ein solche Instanz ist eine endliche Folge \((u_1,v_1),\dots,(u_n,v_n)\) von Paaren in I, so dass 
    \[p=(u_1,v_1)\text{ und }u_1\dots u_n=v_1\dots v_n\]
    Gibt es ein Lösung für eine Instanz des modifizierten Postschen Korrespondenzproblems, so heißt diese Instanz lösbar.\\
    Das modifizierte Postsche Korrespondenzproblem über einem Alphabet \(\Sigma\), kurz \(MPCP_\Sigma\) ist die Menge aller 
    lösbarer Instanzen des modifizierten Postschen Korrespondenzproblems über \(\Sigma\).
\subsection{Fixpunkt}
    Ein Fixpunkt einer berechenbaren Funktion \(f:\mathbb{N}_0\to \mathbb{N}_0\) ist ein \(e\in\mathbb{N}_0\) mit 
    \(\Phi_{f(e)}=\Phi_e\).
\subsection{Rekursionstheorem}
    Für alle partiell berechenbaren Funktion \(\varphi:\mathbb{N}_0^2\rightsquigarrow\mathbb{N}_0\) gibt es ein 
    \(e\in\mathbb{N}_0\) mit \(\Phi_e(x)=\varphi(e,x)\forall x\in\mathbb{N}_0\).
\subsection{Indexmenge}
    Eine Teilmenge \(I\subseteq\mathbb{N}_0\) heißt Indexmenge, wenn \(e\in I\Leftrightarrow e'\in I\)
    für alle \(e,e'\in\mathbb{N}_0\) mit \(\Phi_e=\Phi_{e'}\) gilt.